\newacronym{IT_IOT}{IOT}{Internet of things}
\newacronym{IT_PCB}{kretskort}{PCB}
\newglossaryentry{IT_utlegg}
{
    name=Utlegg,
    text=utlegg,
    description={Utlegg er filene og designet av mønsterkortet}
}
\newglossaryentry{IT_kretskort}
{
    name=Kretskort,
    text=kretskort,
    description={ 
        (PCB) : "Innen elektronikk brukes kretskort til montering av elektroniske komponenter. 
        Foruten å samle og holde komponentene fast er kretskortets oppgave å lage elektriske forbindelser mellom komponentenes bein."\autocite{gls_IT_pcb}   }
}

\newglossaryentry{IT_ruting}
{
    name=Ruting,
    text=ruting,
    description={
        ruting beskriver prosessen å bestemmme hvor og hvordan de elektriske koblingene i kretskortdesignet skal legges}
}


\newglossaryentry{IT_linter}
{
    name=Linter,
    text=linter,
    description={
        sitat: "Lint is the computer science term for a static code analysis tool used to flag programming errors, bugs, stylistic errors and suspicious constructs.
        The term originates from a Unix utility that examined C language source code. A program which performs this function is also known as a linter."\autocite{gls_IT_linter}}
}

\newglossaryentry{IT_LTSpice}
{
    name=LTSpice,
    description={
        LTSpice er et "Simulation Program with Integrated Circuit Emphasis" (SPICE) program utviklet av Analog devices for å simulere elektroniske kretser}
}